\section{Testing and Debugging}
This section covers the testing framework and some debugging tricks of
self-propelled instrumentation.
\subsection{Testing}
We rely on Google C++ Testing Framework for testing, whose source code is
included in self-propelled instrumentation's project directory, that is,
\$SP\_DIR/src/test/gmock.

There are two types of tests in the project, unit tests and system tests.

Each source code file is associated with a unit test suite, consisting of a set
of test cases to test component units.
The unit test suite code is located within the same folder of the source code it
tests (mostly in \$SP\_DIR/src/agent, \$SP\_DIR/src/injector, and
\$SP\_DIR/src/common), and the unit test suite file name has a suffix
``_unittest''.
For example, if a source code file name is agent.cc, then its unit test suite
file name is agent_unittest.cc.

System tests are located in \$SP\_DIR/src/test, which test the complete system.
There are application programs for testing.
The application programs that are written by self-propelled instrumentation
developers are located in \$SP\_DIR/src/test/mutatee.
Those that are real external programs (e.g., unix core utilities) are downloaded
by the script in \$SP\_DIR/scripts/build_mutatees.sh that is invoked implicitly
by Makefile.

To run tests, the current working directory need to be the building directory
\$SP\_DIR/\$PLATFORM, and we invoke test/\$TEST_NAME.exe to run a test.
For example, we can run test/coreutils_systest.exe to run the system test for
instrumenting unix core utilities, or run test/injector_unittest.exe to run the system test for 

\subsection{Debugging}
% Two type of tests
% 1. Unit test
% 2. system test

